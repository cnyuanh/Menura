\nonstopmode{}
\documentclass[letterpaper]{book}
\usepackage[times,inconsolata,hyper]{Rd}
\usepackage{makeidx}
\usepackage[utf8]{inputenc} % @SET ENCODING@
% \usepackage{graphicx} % @USE GRAPHICX@
\makeindex{}
\begin{document}
\chapter*{}
\begin{center}
{\textbf{\huge Package `menura'}}
\par\bigskip{\large \today}
\end{center}
\begin{description}
\raggedright{}
\inputencoding{utf8}
\item[Title]\AsIs{Fitting (Non)-Gaussian diffusion models to phylogenies}
\item[Version]\AsIs{0.4.1}
\item[Description]\AsIs{Fits user-defined stochastic diffusion models to univariate trait data on a phylogeny. Also fits three canned models: Ornstein Uhlenbeck (OU; Gaussian), Cox, Ingersoll, Ross (CIR; Non-Gaussian) and an OU-like model with a Beta stationary distribution (Beta, Non-Gaussian). Models are fitted and parameters are estimated using a Data Augmentation - Metropolis Hastings algorithm. Output can be analysed using the 'coda' package or other appropriate packages for Bayesian analysis and visualisation. Menura is the genus name for the Australian Superb Lyrebird (Menura novaehollandiae), known for being the world's largest passerine, its elaborate tail and its talent for mimicry.}
\item[Author]\AsIs{Simone Blomberg }\email{s.blomberg1@uq.edu.au}\AsIs{}
\item[Maintainer]\AsIs{Simone Blomberg }\email{s.blomberg1@uq.edu.au}\AsIs{}
\item[Depends]\AsIs{R (>= 3.2.0)}
\item[License]\AsIs{GPL-3}
\item[Encoding]\AsIs{UTF-8}
\item[LazyData]\AsIs{true}
\item[Imports]\AsIs{ape, sde, stats, graphics}
\item[NeedsCompilation]\AsIs{no}
\end{description}
\Rdcontents{\R{} topics documented:}
\inputencoding{utf8}
\HeaderA{fit\_model}{Bayesian Estimator of Parameters of Univariate Diffusion Models for Continuous Trait Evolution}{fit.Rul.model}
\methaliasA{fit\_model.default}{fit\_model}{fit.Rul.model.default}
%
\begin{Description}\relax
This function estimates posterior distributions for evolutionary models of continuous traits on a phylogeny. The evolutionary processes considered here belong to a class of diffusion processes which are typically given as solutions to the stochastic differential equations of the form given by
\deqn{dX_t = a(X_t, alpha, mu) dt + b(X_t, sigma) dW_t,~~X_0 = x_0}{}
where \eqn{X_t}{} denote the state variable (ie the trait), \code{t} the time,
the drift function \code{a} and
the diffusion function \code{b} are known
in parametric from where
\code{alpha}, \code{mu}, and \code{sigma}
are the parameters, and \eqn{W_t}{} is Brownian motion.
The value of \eqn{X_t}{} at time \eqn{t_0}{}, \eqn{X_0}{},
is independent of the \eqn{W_t}{}.
\end{Description}
%
\begin{Usage}
\begin{verbatim}
fit_model(tr, tipdata, rt_value, model, ...)

## Default S3 method:
fit_model(tr, tipdata, rt_value = mean(tipdata),
  model = "OU",
  priors = list(
    alpha = list (df =  function(x, a = 1, b = 125, log_scale = TRUE) {
                            dunif(x, min = a, max = b, log = log_scale)},
                  rf = function(n, a = 1, b = 125) {
                            runif(n, min = a, max = b)} ),
    mu = list (df = function(x, a = 0, b = 20, log_scale = TRUE) {
                            dnorm(x, mean = a, sd = b, log = log_scale)},
               rf = function(n, a = 0, b = 20) {
                            rnorm(n, mean = a, sd = b)}),
    sigma = list (df = function(x, a = 1, b = 225, log_scale = TRUE) {
                            dunif(x, min = a, max = b, log = log_scale) },
                   rf = function(n, a = 1, b = 225) {
                            runif(n, min = a, max = b)})
  ),
  proposals = list(
    alpha = list (df = function(n, alpha, gamma = 0.5, log_scale = TRUE) {
                    dlnorm(n, meanlog = log(alpha), sdlog = gamma,
                    log = log_scale) },
                  rf = function(n, alpha, gamma = 0.5) {
                    rlnorm(n, meanlog = log(alpha), sdlog = gamma) }),
    mu = list (df = function(n, mu, gamma = 0.5, log_scale = TRUE) {
                    dnorm(n, mean = mu, sd = gamma, log = log_scale)},
               rf = function(n, mu, gamma = 0.5) {
                    rnorm(n, mean = mu, sd = gamma)}),
    sigma = list(df = function(n, sigma, gamma = 0.5, log_scale = TRUE) {
                    dlnorm(n, meanlog = log(sigma), sdlog = gamma,
                    log = log_scale)},
                  rf = function(n, sigma, gamma = 0.5) {
                    rlnorm(n, meanlog = log(sigma), sdlog = gamma)})
  ),
  mcmc_type = "tanner-wong", alpha = NULL, mu = NULL, sigma = NULL,
  N=1000, init_method="sim", update_method="subtree", iters=5000,
  method = "euler", ...)
\end{verbatim}
\end{Usage}
%
\begin{Arguments}
\begin{ldescription}
\item[\code{tr}] single evolutionary tree as an object of the 'phylo' class in
the \code{ape} R package.

\item[\code{tipdata}] a numeric vector containing values of the trait at the tip.
These must be in the same order as those in the tr\$tip.label

\item[\code{rt\_value}] value of the trait at the root.

\item[\code{model}] either a list containing drift and diffusion coefficients
in quote format as functions of alpha, mu and sigma,
or a string ("OU", "CIR", or "Beta") specifying diffusion process.
See Details.

\item[\code{priors}] list of lists containing functions for prior distributions
of the model parameters.

\item[\code{proposals}] list of lists containing functions for proposal distributions
of the model parameters.

\item[\code{mcmc\_type}] Type of MCMC algorithm

\item[\code{alpha}] NULL if alpha is to be estimated, else either a numeric
value of a numeric vector specifying the value of the parameter for
all the branches/edges. In the latter case, the values must be
specifying in order of the edges in the \code{tr} object.

\item[\code{mu}] same as \code{alpha}

\item[\code{sigma}] same as \code{alpha}

\item[\code{N}] data augmentation frequency.

\item[\code{init\_method}] method for initial data imputation.
Currently only the \code{"sim"} option is available.

\item[\code{update\_method}] method for data imputation during the MCMC.
The \code{"subtree"} will only update a random part of the tree at each
iteration, where as the option \code{"tree"} will update the whole tree. See Details

\item[\code{iters}] number of MCMC iterations.

\item[\code{method}] Numerical approximation method to use.

\item[\code{...}] further arguments for future extensions.
\end{ldescription}
\end{Arguments}
%
\begin{Details}\relax
Given the root and tip values, the tree, drift and diffusion functions, the Data Augmentation - Markov Chain Monte Carlo (DA-MCMC) estimates of the parameters are obtained. Parameters may be the same across the tree or allowed to differ in different parts of the tree.

Due to the low frequency nature of the data, we employ Data Augmentation of the evolutionary trajectory (the 'fossil record'), effectively imputing the missing trajectory, and updating trajectories at each MCMC iteration.

If the diffusion process considered is the Ornstein-Uhlenbeck (OU) process, Cox-Ingersoll-Ross (CIR) process or Beta process (Beta) then this can be specified by setting the \code{model} to "OU", "CIR", or "Beta", respectively. In the case of a user-defined diffusion process, the estimation of model parameters can be done by specifying drift and diffusion coefficients in a list assigned to \code{model}. In this case, the list object \code{model} must include functions \code{d} and \code{s} which are functions of \code{t}, \code{x} and \code{theta} which are the time variable, the space variable, and a vector consisting of the parameter values \code{alpha}, \code{mu} and \code{sigma}, and drift coefficient as a list containing a \code{quote} and diffusion coefficient as a list objects which include \code{diffusion} as the diffusion coefficient and \code{x} as the first derivatives of diffusion coefficients.

The Ornstein–Uhlenbeck (OU) model is given by \deqn{dX_t = alpha (mu - X_t) dt + sigma~ dW_t,}{} with \eqn{X_0 = x_0 > 0}{}, \eqn{W_t}{} is Brownian motion, alpha, mu, and sigma are the model parameters where alpha and sigma are positive values.

The Cox-Ingersoll-Ross (CIR) model is given by \deqn{dXt = alpha (mu -
  X_t) dt + sigma~sqrt(X_t) dWt,}{} with \eqn{X_0 = x_0 > 0}{}, where Wt is
Brownian motion, alpha, mu, and sigma are the model parameters which
are all positive values. If the \code{model == "CIR"} is specified, then the parameter estimation is done by using the transformation \eqn{Y = sqrt(X)}{} of the Ito diffusion process.

The Beta model is given by \deqn{ dX_t = alpha(mu - X_t) dt + sigma~sqrt(X_t (1-X_t)) dW_t,}{} with \eqn{X_0 = x_0 > 0}{}, \eqn{W_t}{} is Brownian motion, alpha, mu, and sigma are the model parameters where alpha and sigma are positive values. If the \code{model ==} "Beta" is specified, then the parameter estimation is done by using transformation \eqn{Y = 2~sin^{-1}(X)}{} of the Ito diffusion process.
\end{Details}
%
\begin{Section}{Methods (by class)}
\begin{itemize}

\item \code{default}: Bayesian Estimator of Diffusion Process

\end{itemize}
\end{Section}
%
\begin{Examples}
\begin{ExampleCode}
set.seed(1)
rpkgs <- c("sde", "ape", "msm")
lapply(rpkgs, require, character.only = TRUE)
# Number of tips
ntips <- 128
# SDE parameters
true.alpha <- 10
true.mu <- 5
true.sigma <- 2
t.root.value <- true.mu
iters <- 200
# Generate tip values
set.seed(1)
tr <-  compute.brlen(stree(n=ntips, type="balanced"))
f_TrCir <- function(x, l)
  rcCIR(n=1, Dt=l, x0=x, theta=c(true.alpha*true.mu, true.alpha, true.sigma))
t.tipdata <- rTraitCont(tr, f_TrCir, ancestor = FALSE, root.value = t.root.value)

set.seed(1)
model.1 <- fit_model(tr=tr, tipdata=t.tipdata, rt_value=t.root.value, iters=iters,
                model = "CIR", alpha = 10,  mu = 15, sigma = NULL,
                N=10, init_method = "sim", update_method = "subtree")

# Look at the MCMC trace of the parameters
# summary(model.1)
model.1

## Use the coda package to analyse the mcmc chain
# library(coda)
# plot(model.1$mcmctrace)
# summary(model.1$mcmctrace)

## The same can be done using the user-defined specification:
model <- list()
model$d <- function (t, x, theta) {
  ((theta[1]*theta[2] - 0.25* theta[3]^2) / (2 * x)) - theta[2] * x / 2
}
model$s <- function(t, x, theta) {
  0.5 * theta[3]
}
model$drift <- quote(((alpha * mu - 0.25 * sigma^2) / (2*x)) -
                      alpha * x / 2)
model$diffusion <- quote(sigma/2)
model$dx_diffusion <- quote(0)

tipdata  <- sqrt(t.tipdata)
rt_value <- sqrt(t.root.value)
set.seed(1)
model.2 <- fit_model(tr=tr, tipdata=tipdata, rt_value=rt_value, iters=iters,
                model = model, alpha = 10,  mu = 15, sigma = NULL,
                N=10, init_method = "sim", update_method = "subtree")

\end{ExampleCode}
\end{Examples}
\inputencoding{utf8}
\HeaderA{phylo\_sde}{Simulate a CIR Diffusion Process in the Tree of Life}{phylo.Rul.sde}
%
\begin{Description}\relax
Starting from the root of the tree - which is assumed to start at
time 0, and the root value is known - a recursive scheme is used in
simulating a CIR process in the branches of the tree.
\end{Description}
%
\begin{Usage}
\begin{verbatim}
phylo_sde(tr, rt_value, N, theta, model, method = "euler", ...)
\end{verbatim}
\end{Usage}
%
\begin{Arguments}
\begin{ldescription}
\item[\code{tr}] a modified object of class \code{phylo} as in the
\code{ape} R package. In this version, the CIR process
parameters alpha, mu and sigma for each of the branch
is included as vectors in the same order
as the edge (branch) labelling.

\item[\code{rt\_value}] value at the root.

\item[\code{N}] data imputation frequency.

\item[\code{theta}] matrix of parameter values for each edge of the tree.

\item[\code{model}] a list containing drift, diffusion
and their partial differentiation as quotes. For the Euler scheme
the drift coefficient as \code{drift}, the diffusion coefficient as
\code{diffusion}, and the partial differentiation of
\code{diffusion} by \code{x} as \code{dx\_diffusion} is required.
See the Examples.

\item[\code{method}] currently only the "euler" scheme is used.

\item[\code{...}] not used.
\end{ldescription}
\end{Arguments}
%
\begin{Details}\relax
The number of samples imputed in a branch (edge) is proportional to the length
of the branch. First, the samples are imputed for the two root edges.
The end points of these are taken to be the starting points of
the successive branches. This process is done recursive for until
the tip nodes are reached.

The number of samples imputed on a branch is equal to
\code{round(N * branch\_length)}.
As, the next branch starts from the end point of the previous,
the branch start and stop times would change,
which would depend on the data imputation frequency N.

In case, if length of an edge is small, no samples may be imputed for
such a branch. As such, the simulated output may contain branches
of zero length. This can be avoided by employing higher value for N.
\end{Details}
%
\begin{Value}
lst a list of length equal to the number of
branches in the object \code{tr}. Elements of lst are time series
objects which are the simulated paths.
\end{Value}
%
\begin{Examples}
\begin{ExampleCode}
set.seed(1)
rpkgs <- c("sde", "ape", "msm")
lapply(rpkgs, require, character.only = TRUE)
# Number of tips
# Random tree with 64 tips
tr <-  compute.brlen(rtree(n=64))

# SDE parameters
Nedges <- length(tr$edge.length)
dclade <- max(which(tr$edge[,1] == tr$edge[1,1])) - 1
alpha <- mu <- sigma <- rep(0, Nedges)
alpha[1:Nedges]  <- 0.1
mu[1:Nedges] <- 0
sigma[1:Nedges] <- 1
rt_value <- 0
tipdata <- rTraitCont(tr, "OU", sigma=sigma, alpha=alpha, theta=mu,
                       root.value=rt_value)
model <- list()
model$d <- function (t, x, theta) {
  theta[1] * (theta[2] - x)
}
model$s <- function(t, x, theta) {
  theta[3]
}
model$drift <- quote(alpha * (mu - x))
model$diffusion <- quote(sigma)
model$dx_diffusion <- quote(0)
theta <- cbind(alpha=alpha, mu=mu, sigma=sigma)
N <- 100
lst <- phylo_sde (tr=tr, rt_value=rt_value, theta=theta, model=model,
                   N=N, method="euler")
\end{ExampleCode}
\end{Examples}
\inputencoding{utf8}
\HeaderA{tree\_logL}{Calculates the Log Likelihood of the Diffusion Process in a Tree of Life}{tree.Rul.logL}
%
\begin{Description}\relax
Euler approximated Log likelihood of the diffusion process in the tree.
\end{Description}
%
\begin{Usage}
\begin{verbatim}
tree_logL(fossils, tr, tipdata, lst, alpha, mu, sigma, model, method, ...)
\end{verbatim}
\end{Usage}
%
\begin{Arguments}
\begin{ldescription}
\item[\code{fossils}] Name of nodes that are fossils.
\item[\code{tr}] object of the class "phylo" as in the \code{ape} R package which
contains the information of the tree structure.

\item[\code{tipdata}] a numeric vector containing tip values in the
same order of the tip labels in \code{tr\$tip.label}.

\item[\code{lst}] list containing the diffusion paths of the \code{tr} object in the same
order as the edges in the \code{tr} object.

\item[\code{alpha}] vector containing the parameter value of theta\_1 for each edge \code{tr} object,
and the parameters location corresponds to the same edge numbering in the \code{tr} object.

\item[\code{mu}] similar to definition of \code{alpha}.

\item[\code{sigma}] similar to definition of \code{alpha}.

\item[\code{model}] a list which contains functions \code{d}, \code{s} and, possibly, \code{s\_x} which are drift component, diffusion component, and partial derivative of diffusion component of the diffusion process.


\item[\code{method}] Numerical method apprximate the sde.

\item[\code{...}] not used.
\end{ldescription}
\end{Arguments}
%
\begin{Value}
logL a number.
\end{Value}
%
\begin{Examples}
\begin{ExampleCode}
set.seed(1)
rpkgs <- c("sde", "ape", "msm")
lapply(rpkgs, require, character.only = TRUE)
# Number of tips
# Random tree with 64 tips
tr <-  compute.brlen(rtree(n=64))

# SDE parameters
Nedges <- length(tr$edge.length)
dclade <- max(which(tr$edge[,1] == tr$edge[1,1])) - 1
alpha <- mu <- sigma <- rep(0, Nedges)
alpha[1:Nedges]  <- 0.1
mu[1:Nedges] <- 0
sigma[1:Nedges] <- 1
rt_value <- 0
tipdata <- rTraitCont(tr, "OU", sigma=sigma, alpha=alpha, theta=mu,
                       root.value=rt_value)
model <- list()
model$d <- function (t, x, theta) {
  theta[1] * (theta[2] - x)
}
model$s <- function(t, x, theta) {
  theta[3]
}
model$drift <- quote(alpha * (mu - x))
model$diffusion <- quote(sigma)
model$dx_diffusion <- quote(0)
theta <- cbind(alpha=alpha, mu=mu, sigma=sigma)
N <- 100
lst <- phylo_sde (tr=tr, rt_value=rt_value, theta=theta, model=model,
                   N=N, method="euler")

loglike <-  tree_logL (tr=tr, tipdata=tipdata, lst=lst,
                              alpha=theta[, "alpha"],
                              mu=theta[, "mu"],
                              sigma=theta[, "sigma"], model,
                              method = "euler")

\end{ExampleCode}
\end{Examples}
\printindex{}
\end{document}
